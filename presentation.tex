\documentclass{beamer}

\makeatletter
\def\input@path{{Padova/}}
\makeatother

\usetheme{Padova}

\graphicspath{ {images/} }

\title{Migliorare la Business Intelligence con l'AI}
\subtitle{Discussione di Laurea Triennale in Informatica}
\author{Riccardo Stefani}
\date{23 Luglio 2025}


\begin{document}

	\maketitle

	\begin{frame}{Indice}
		\tableofcontents
	\end{frame}


	\section{Introduzione}

	\begin{frame}{Introduzione - Oribea}
		\textbf{Startup innovativa} fondata nel \textbf{2024} a \textbf{San Marino}

		\begin{columns}
			\begin{column}{0.5\textwidth}
				\begin{block}{Mission}
					\begin{itemize}
						\item Soluzioni \textbf{AI avanzate}
						\item Migliorare \textbf{efficienza aziendale}
						\item Focus su \textbf{LLM}
						\item \textbf{Agenti intelligenti}
					\end{itemize}
				\end{block}
			\end{column}
			\begin{column}{0.5\textwidth}
				\begin{block}{Prodotti principali}
					\begin{itemize}
						\item Automazione \textbf{processi aziendali}
						\item \textbf{AI Task Builder}
						\item \textbf{AI Chatbot Builder}
					\end{itemize}
				\end{block}
			\end{column}
		\end{columns}

		\begin{figure}[h]
			\centering
			\includegraphics[width=0.5\textwidth]{oribea-logo.png}
		\end{figure}
	\end{frame}

	\begin{frame}{Introduzione - Il progetto}
		\textbf{Scopo}: Automatizzare la \textbf{Business Intelligence} per e-commerce

		\begin{columns}
			\begin{column}{0.5\textwidth}
				\begin{block}{Analisi delle vendite}
					\begin{itemize}
						\item \textbf{Input:} Elenco di vendite di e-commerce
						\item \textbf{Output:} Report automatico con statistiche e grafici
					\end{itemize}
				\end{block}
			\end{column}
			\begin{column}{0.5\textwidth}
				\begin{block}{Sistema di raccomandazione}
					\begin{itemize}
						\item \textbf{Input:} Storico acquisti clienti-prodotti, e nomi significativi dei prodotti
						\item \textbf{Output:} Raccomandazioni personalizzate di prodotti e clienti
					\end{itemize}
				\end{block}
			\end{column}
		\end{columns}

		\begin{alertblock}{Motivazione personale}
			Approfondire soluzioni \textbf{AI} per \textbf{analisi dati} e \textbf{sistemi di raccomandazione}
		\end{alertblock}
	\end{frame}


	\section{Analisi delle vendite}

	\begin{frame}{Analisi delle vendite - L'idea}
		L'idea di \textbf{analisi delle vendite} prevede di fornire un sistema automatizzato che generi:

		\begin{columns}
			\begin{column}{0.6\textwidth}
				\begin{itemize}
					\item \textbf{Statistiche utili}
					\item \textbf{Grafici}
					\item \textbf{Resoconto dell'andamento} tramite LLM
				\end{itemize}
			\end{column}
			\begin{column}{0.4\textwidth}
				\begin{figure}
					\centering
					\includegraphics[width=\textwidth]{Oribea - Esempio di report delle vendite.png}
				\end{figure}
			\end{column}
		\end{columns}

		\vspace{0.5em}
		Oribea ha fornito un \textbf{prototipo} di analisi delle vendite compiuta manualmente, visibile in figura, come base per lo sviluppo della soluzione automatizzata.
	\end{frame}

	\begin{frame}{Analisi delle vendite - Pipeline per la\\ generazione del report}
		\textbf{Pipeline di elaborazione} per l'analisi automatizzata dei dati di vendita e la generazione del report:

		\begin{figure}
			\centering
			\includegraphics[width=\textwidth]{Diagramma pipeline analisi delle vendite.png}
		\end{figure}
	\end{frame}

	\begin{frame}{Analisi delle vendite - Formati di output\\ del report}
		\textbf{Output dell'analisi}: Il report generato viene presentato all'utente in \textbf{multipli formati}:

		\begin{columns}
			\begin{column}{0.33\textwidth}
				\begin{block}{PDF}
					Formato \textbf{professionale} per archiviazione e condivisione
				\end{block}
			\end{column}
			\begin{column}{0.33\textwidth}
				\begin{block}{HTML}
					Visualizzazione \textbf{interattiva} e responsiva nel browser
				\end{block}
			\end{column}
			\begin{column}{0.33\textwidth}
				\begin{block}{Email}
					\textbf{Invio automatico} del report alla mail dell'utente
				\end{block}
			\end{column}
		\end{columns}

		\begin{figure}
			\centering
			\includegraphics[width=\textwidth]{Diagramma formati output del report.png}
		\end{figure}
	\end{frame}


	\section{Sistema di raccomandazione}

	\begin{frame}{Sistema di raccomandazione - L'idea}
		\begin{columns}
			\begin{column}{0.6\textwidth}
				\begin{block}{Obiettivo}
					Combinare i vantaggi del \textbf{Collaborative Filtering} con quelli della \textbf{similarità basata su contenuto} per ottenere raccomandazioni più accurate e robuste.
				\end{block}

				\begin{block}{Approccio}
					Implementazione di un algoritmo di \textbf{Rank Fusion}, utilizzando in particolare il \textbf{Reciprocal Rank Fusion (RRF)}, per combinare efficacemente i risultati dei due sistemi.
				\end{block}
			\end{column}
			\begin{column}{0.4\textwidth}
				\begin{figure}
					\centering
					\includegraphics[width=\textwidth]{Diagramma Pipeline sistema di raccomandazione.png}
				\end{figure}
			\end{column}
		\end{columns}
	\end{frame}

	\begin{frame}{Sistema di raccomandazione -\\ Collaborative Filtering}
		Un sistema di \textbf{Collaborative Filtering} sfrutta i comportamenti passati degli utenti per generare raccomandazioni.

		\begin{columns}
			\begin{column}{0.6\textwidth}
				\begin{block}{Principio base}
					\begin{itemize}
						\item Analizza \textbf{preferenze utenti simili}
						\item Identifica \textbf{pattern di acquisto}
						\item Predice \textbf{nuovi interessi}
					\end{itemize}
				\end{block}
			\end{column}
			\begin{column}{0.38\textwidth}
				\begin{figure}
					\centering
					\includegraphics[width=\textwidth]{Collaborative-Filtering.png}
				\end{figure}
			\end{column}
		\end{columns}

		\begin{alertblock}{Logica}
			"Gli utenti con \textbf{gusti simili} nel passato avranno \textbf{preferenze simili} in futuro"
		\end{alertblock}
	\end{frame}

	\begin{frame}{Sistema di raccomandazione - Similarità}
		Un sistema di \textbf{Similarità} sfrutta le caratteristiche semantiche dei prodotti per generare raccomandazioni.

		\begin{columns}
			\begin{column}{0.6\textwidth}
				\begin{block}{Principio base}
					\begin{itemize}
						\item Analizza \textbf{nomi descrittivi} dei prodotti
						\item Calcola \textbf{Cosine Similarity}
						\item Identifica \textbf{prodotti simili}
					\end{itemize}
				\end{block}
			\end{column}
			\begin{column}{0.5\textwidth}
				\begin{figure}
					\centering
					\includegraphics[width=\textwidth]{Cosine-Similarity.png}
				\end{figure}
			\end{column}
		\end{columns}

		\begin{alertblock}{Logica}
			"Se due prodotti hanno \textbf{caratteristiche simili}, le raccomandazioni per uno possono essere \textbf{utili anche per l'altro}"
		\end{alertblock}
	\end{frame}

	\begin{frame}{Sistema di raccomandazione - Formato di archiviazione delle matrici}
        \textbf{Confronto} di \textbf{5 formati} di archiviazione per ottimizzare la lettura:

		\begin{columns}
			\begin{column}{0.6\textwidth}
				\begin{block}{Formati analizzati}
					\begin{itemize}
						\item \textbf{CSV}: Semplice ma inefficiente
						\item \textbf{HDF5}: Form. binario strutturato
						\item \textbf{NPY}: Nativo NumPy veloce
						\item \textbf{Parquet}: Colonnare compresso
						\item \textbf{Zarr}: Array n-dimensionali \textbf{cloud}
					\end{itemize}
				\end{block}
			\end{column}
			\begin{column}{0.4\textwidth}
				\begin{figure}
					\centering
					\includegraphics[width=\textwidth]{time_vs_memory.png}
				\end{figure}
			\end{column}
		\end{columns}

		\begin{alertblock}{Risultato}
			\textbf{Zarr} offre il miglior \textbf{compromesso} tra velocità e memoria
		\end{alertblock}
	\end{frame}

	\begin{frame}{Sistema di raccomandazione - Rank Fusion}
		Il \textbf{Reciprocal Rank Fusion} combina efficacemente i ranking di diversi sistemi di raccomandazione:

		\begin{columns}
			\begin{column}{0.6\textwidth}
				\begin{block}{Processo RRF}
					\begin{itemize}
						\item \textbf{Input}: Ranking da CF e Similarità
						\item \textbf{Calcolo}: Score reciproco per posizione
						\item \textbf{Fusione}: Somma dei punteggi
						\item \textbf{Output}: Ranking finale unificato
					\end{itemize}
				\end{block}
			\end{column}
			\begin{column}{0.4\textwidth}
				\begin{figure}
					\centering
					\includegraphics[width=\textwidth]{Reciprocal-Rank-Fusion.png}
				\end{figure}
			\end{column}
		\end{columns}

		\begin{alertblock}{Svantaggio}
			Non considera i punteggi nei ranking di input, solo la loro posizione
		\end{alertblock}
	\end{frame}

	\begin{frame}{Sistema di raccomandazione - Metriche di valutazione}
		Saepe eveniet ut et voluptates repudiandae sint et molestiae non recusandae.

		\begin{columns}
			\begin{column}{0.5\textwidth}
				\begin{block}{Precisione}
					$$ Precision@k = \frac{|R_k \cap T|}{k} $$
				\end{block}
			\end{column}
			\begin{column}{0.5\textwidth}
				\begin{block}{Recall}
					$$ Recall@k = \frac{|R_k \cap T|}{|T|} $$
				\end{block}
			\end{column}
		\end{columns}
	\end{frame}

	\begin{frame}{Sistema di raccomandazione - Explainability}
		Itaque earum rerum hic tenetur a sapiente delectus, ut aut reiciendis voluptatibus maiores alias consequatur.

		\begin{itemize}
			\item Interpretabilità delle raccomandazioni
			\item Trasparenza degli algoritmi
			\item Giustificazione delle scelte
		\end{itemize}
	\end{frame}


	\section{Deploy e frontend}

	\begin{frame}{Deploy e frontend - Integrazione con\\ Google Cloud}
		Aut perferendis doloribus asperiores repellat. Hic tenetur a sapiente delectus.

		\begin{block}{Servizi utilizzati}
			\begin{itemize}
				\item Google Cloud Functions
				\item Cloud Storage
				\item Cloud SQL
				\item Cloud Scheduler
			\end{itemize}
		\end{block}
	\end{frame}

	\begin{frame}{Deploy e frontend - Interfacce frontend}
		Ut aut reiciendis voluptatibus maiores alias consequatur aut perferendis doloribus asperiores repellat.

		\begin{columns}
			\begin{column}{0.5\textwidth}
				\begin{block}{Dashboard}
					Interfaccia per la visualizzazione dei dati e delle statistiche
				\end{block}
			\end{column}
			\begin{column}{0.5\textwidth}
				\begin{block}{API}
					Endpoint REST per l'integrazione con sistemi esterni
				\end{block}
			\end{column}
		\end{columns}

		\begin{figure}
			\centering
			\begin{minipage}{0.4\textwidth}
				\centering
				\includegraphics[width=\textwidth]{Frontend Sales Analysis.png}
			\end{minipage}
			\hspace{0.05\textwidth}
			\begin{minipage}{0.2\textwidth}
				\centering
				\includegraphics[width=\textwidth]{Frontend Recommendation.png}
			\end{minipage}
		\end{figure}
	\end{frame}


	\section{Ottimizzazione}

	\begin{frame}{Ottimizzazione}
		Nam libero tempore, cum soluta nobis est eligendi optio cumque nihil impedit quo minus id quod maxime placeat facere possimus.

		\begin{itemize}
			\item Ottimizzazione delle performance
			\item Riduzione dei costi computazionali
			\item Miglioramento dell'accuratezza
			\item Scalabilità del sistema
		\end{itemize}

		\begin{exampleblock}{Risultati}
			Omnis voluptas assumenda est, omnis dolor repellendus. Temporibus autem quibusdam et aut officiis debitis.
		\end{exampleblock}
	\end{frame}


	\section{Conclusioni e considerazioni finali}

	\begin{frame}{Conclusioni e considerazioni finali}
		Sed ut perspiciatis unde omnis iste natus error sit voluptatem accusantium doloremque laudantium, totam rem aperiam, eaque ipsa quae ab illo inventore veritatis et quasi architecto beatae vitae dicta sunt explicabo.

		\begin{block}{Risultati ottenuti}
			\begin{itemize}
				\item Miglioramento significativo delle performance del sistema
				\item Riduzione dei tempi di elaborazione
				\item Aumento dell'accuratezza delle raccomandazioni
				\item Scalabilità dimostrata in ambiente di produzione
			\end{itemize}
		\end{block}

		\begin{alertblock}{Sviluppi futuri}
			Nemo enim ipsam voluptatem quia voluptas sit aspernatur aut odit aut fugit, sed quia consequuntur magni dolores eos qui ratione voluptatem sequi nesciunt.
		\end{alertblock}
	\end{frame}

	\begin{frame}{Domande?}
		\begin{center}
			\Large
			\textbf{Grazie per l'attenzione!}
			
			\vspace{1em}
			
			\normalsize
			\textit{Presentazione disponibile su:}\\
			\url{https://github.com/Rickyz03/Presentazione-Discussione-LT-Informatica}
			
			\vspace{2em}
			
			\Large
			\textbf{Ci sono domande?}
		\end{center}
	\end{frame}


\end{document}
